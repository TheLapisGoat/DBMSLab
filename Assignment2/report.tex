\documentclass[7pt]{article}
\usepackage{graphicx} % Required for inserting images
\usepackage{amsmath}
\usepackage[T1]{fontenc}

%Decrease margins
\usepackage[margin=0.5in]{geometry}

\title{DBMS_ER}
\author{Sourodeep Datta}
\date{January 2024}

%Remove section numbering
\makeatletter
\renewcommand{\@seccntformat}[1]{}
\makeatother

\begin{document}

{\centering
    \textbf{\LARGE{Report}} \\
}
\vspace{0.5cm}

\section{\Large{Schema}}

\begin{center}
\begin{tabular}{|p{8cm} | p{10cm}|}
    \hline
    \vspace{2pt} Schemas & \vspace{2pt} Foreign Keys \\
    \vspace{2pt}
    \textit{college(\underline{name}, {location})} & \vspace{2pt} -\\
    \vspace{2pt}
    \textit{participant(\underline{pid}, name)} & \vspace{2pt} -\\
    \vspace{2pt}
    \textit{volunteer(\underline{roll})} & \vspace{2pt} -\\
    \vspace{2pt}
    \textit{event(\underline{eid}, date, ename, type)} & \vspace{2pt} -\\
    \vspace{2pt}
    \textit{student(\underline{roll}, name, dept)} & \vspace{2pt} -\\
    \vspace{2pt}
    \textit{role(\underline{rid}, rname, description)} & \vspace{2pt} -\\
    \vspace{2pt}
    \textit{student\_from(college\_name, \underline{\smash{pid}})} & \vspace{2pt} \text{college\_name -{\textgreater} college\_name, pid -{\textgreater} participant\_pid} \\
    \vspace{2pt}
    \textit{participant\_has(\underline{\smash{pid}}, \underline{\smash{eid}})} & \vspace{2pt} \text{pid -{\textgreater} participant\_pid, eid -{\textgreater} event\_eid} \\
    \vspace{2pt}
    \textit{volunteer\_has(\underline{\smash{roll}}, \underline{\smash{eid}})} & \vspace{2pt} \text{roll -{\textgreater} volunteer\_roll, eid -{\textgreater} event\_eid} \\
    \vspace{2pt}
    \textit{manage(\underline{roll}, \underline{eid})} & \vspace{2pt} \text{roll -{\textgreater} student\_roll, eid -{\textgreater} event\_eid} \\
    \vspace{2pt}
    \textit{student\_has(\underline{roll}, \underline{rid})} & \vspace{2pt} \text{roll -{\textgreater} student\_roll, rid -{\textgreater} role\_rid} \\
    \hline
    \end{tabular}
\end{center}

\begin{flushleft}
\textit{\underline{Underlined} attributes are Primary Keys.}
\end{flushleft}

\subsection{\text{Table \& Attribute Definitions}}
\begin{itemize}
    \item college ( \\
    name varchar(511) primary key, \\
    location varchar(2047) not null \\
    )

    \item participant ( \\
    pid varchar(20) primary key, \\
    name varchar(255) not null \\
    )

    \item volunteer ( \\
    roll varchar(20) primary key\\
    )

    \item event ( \\
    eid varchar(20) primary key,    \\
    date DATE not null,     \\
    ename varchar(255) not null,    \\
    type varchar(255)   \\
    )

    \item student ( \\
    roll varchar(20) primary key,   \\
    name varchar(255) not null, \\
    dept varchar(255) not null \\
    )

\newpage

    \item role (    \\
    rid varchar(20) primary key,       \\
    rname varchar(255) not null,    \\
    description varchar(255)    \\
    )

    \item student\_from (  \\
    college\_name varchar(255),  \\
    pid varchar(20),    \\
    foreign key (college\_name) references college(name),    \\
    foreign key (pid) references participant(pid),  \\
    primary key (college\_name, pid)     \\
    )

    \item participant\_has (     \\
    pid varchar(20),        \\
    eid varchar(20),        \\
    foreign key (pid) references participant(pid),      \\
    foreign key (eid) references event(eid),     \\
    primary key (pid, eid)  \\
    )

    \item volunteer\_has (   \\
    roll varchar(20),   \\
    eid varchar(20),        \\
    foreign key (roll) references volunteer(roll),  \\
    foreign key (eid) references event(eid),        \\
    primary key (roll, eid) \\
    )  

    \item manage (     \\
    roll varchar(20),   \\
    eid varchar(20),    \\
    foreign key (roll) references student(roll),    \\
    foreign key (eid) references event(eid), \\
    primary key (roll, eid) \\
    )

    \item student\_has (     \\
    roll varchar(20),   \\
    rid varchar(20),        \\
    foreign key (roll) references student(roll),    \\
    foreign key (rid) references role(rid),      \\
    primary key (roll, rid) \\
    )
\end{itemize}

\begin{flushleft}
    \textit{Primary keys are implicitly not null.}
\end{flushleft}

\newpage

\section{\Large{Commands}}

\subsection{\text{Create Tables \& Insert Data}}

\begin{itemize}
    \item -- Create college table   \\
CREATE TABLE college (              \\
    name VARCHAR(511) PRIMARY KEY,  \\
    location VARCHAR(2047) NOT NULL \\
);  \vspace{2pt}

-- Insert sample data               \\
INSERT INTO college (name, location) VALUES     \\
    ('IITB', 'Mumbai'), \\
    ('MIT', 'Cambridge'),   \\
    ('Stanford', 'Palo Alto'),  \\
    ('Harvard', 'Cambridge'),   \\
    ('Caltech', 'Pasadena');

    \item -- Create participant table   \\
    CREATE TABLE participant (      \\
        pid VARCHAR(20) PRIMARY KEY,    \\
        name VARCHAR(255) NOT NULL  \\
    );  \vspace{2pt}
    
    -- Insert sample data   \\
    INSERT INTO participant (pid, name) VALUES  \\
        ('P001', 'John Doe'),   \\
        ('P002', 'Jane Smith'), \\
        ('P003', 'Michael Johnson'),    \\
        ('P004', 'Emily Brown'),    \\
        ('P005', 'Robert Davis');

    \item -- Create volunteer table \\
    CREATE TABLE volunteer (    \\
        roll VARCHAR(20) PRIMARY KEY    \\
    );  \vspace{2pt}
    
    -- Insert sample data   \\
    INSERT INTO volunteer (roll) VALUES \\
        ('V001'),   \\
        ('V002'),   \\
        ('V003'),   \\
        ('V004'),   \\
        ('V005');
    
    \item -- Create event table \\
    CREATE TABLE event (    \\
        eid VARCHAR(20) PRIMARY KEY,    \\
        date DATE NOT NULL, \\
        ename VARCHAR(255) NOT NULL,    \\
        type VARCHAR(255)   \\
    );  \vspace{2pt}
    
    -- Insert sample data   \\
    INSERT INTO event (eid, date, ename, type) VALUES   \\
        ('E001', '2024-02-01', 'Megaevent', 'Conference'),  \\
        ('E002', '2024-02-15', 'Tech Symposium', 'Symposium'),  \\
        ('E003', '2024-03-05', 'Sports Fest', 'Sports'),    \\
        ('E004', '2024-04-10', 'Cultural Night', 'Cultural'),   \\
        ('E005', '2024-05-20', 'Workshop Series', 'Workshop'); 

\newpage

    \item -- Create student table   \\
    CREATE TABLE student (              \\
        roll VARCHAR(20) PRIMARY KEY,   \\
        name VARCHAR(255) NOT NULL, \\
        dept VARCHAR(255) NOT NULL  \\
    );  \vspace{2pt}
    
    -- Insert sample data   \\
    INSERT INTO student (roll, name, dept) VALUES   \\
        ('S001', 'Alice Johnson', 'CSE'),   \\
        ('S002', 'Bob Smith', 'ECE'),   \\
        ('S003', 'Charlie Brown', 'Mechanical'),    \\
        ('S004', 'Diana Davis', 'Chemical'),    \\
        ('S005', 'Edward White', 'Civil');

    \item -- Create role table  \\
    CREATE TABLE role ( \\
        rid VARCHAR(20) PRIMARY KEY,    \\
        rname VARCHAR(255) NOT NULL,    \\
        description VARCHAR(255)    \\
    );  \vspace{2pt}
    
    -- Insert sample data   \\
    INSERT INTO role (rid, rname, description) VALUES   \\
        ('R001', 'Secretary', 'Manages administrative tasks'),  \\
        ('R002', 'Treasurer', 'Handles financial matters'), \\
        ('R003', 'President', 'Leads and oversees the organization'),   \\
        ('R004', 'Event Coordinator', 'Plans and coordinates events'),  \\
        ('R005', 'Public Relations Officer', 'Manages public relations and communications');

    \item -- Create student\_from table \\
    CREATE TABLE student\_from ( \\
        college\_name VARCHAR(255), \\
        pid VARCHAR(20),    \\
        PRIMARY KEY (college\_name, pid),   \\
        FOREIGN KEY (college\_name) REFERENCES college(name),   \\
        FOREIGN KEY (pid) REFERENCES participant(pid)   \\
    );  \vspace{2pt}
    
    -- Insert sample data   \\
    INSERT INTO student\_from (college\_name, pid) VALUES \\
        ('IITB', 'P001'),   \\
        ('MIT', 'P002'),    \\
        ('Stanford', 'P003'),   \\
        ('Harvard', 'P004'),    \\
        ('Caltech', 'P005');

    \item -- Create participant\_has table   \\
    CREATE TABLE participant\_has ( \\
        pid VARCHAR(20),    \\
        eid VARCHAR(20),    \\
        PRIMARY KEY (pid, eid), \\
        FOREIGN KEY (pid) REFERENCES participant(pid),  \\
        FOREIGN KEY (eid) REFERENCES event(eid) \\
    );  \vspace{2pt}

    -- Insert sample data   \\
    INSERT INTO participant\_has (pid, eid) VALUES  \\
        ('P001', 'E001'),   \\
        ('P002', 'E002'),   \\
        ('P003', 'E003'),   \\
        ('P004', 'E004'),   \\
        ('P005', 'E005');
    
    \newpage

    \item -- Create volunteer\_has table \\
CREATE TABLE volunteer\_has (    \\
    roll VARCHAR(20),   \\
    eid VARCHAR(20),    \\
    PRIMARY KEY (roll, eid),    \\
    FOREIGN KEY (roll) REFERENCES volunteer(roll),  \\
    FOREIGN KEY (eid) REFERENCES event(eid) \\
);  \vspace{2pt}

-- Insert sample data   \\
INSERT INTO volunteer\_has (roll, eid) VALUES   \\
    ('V001', 'E001'),   \\
    ('V002', 'E002'),   \\
    ('V003', 'E003'),   \\
    ('V004', 'E004'),   \\
    ('V005', 'E005');

    \item -- Create manage table    \\
    CREATE TABLE manage (   \\
        roll VARCHAR(20),   \\
        eid VARCHAR(20),    \\
        PRIMARY KEY (roll, eid),    \\
        FOREIGN KEY (roll) REFERENCES student(roll),    \\
        FOREIGN KEY (eid) REFERENCES event(eid) \\
    );  \vspace{2pt}
    
    -- Insert sample data   \\
    INSERT INTO manage (roll, eid) VALUES   \\
        ('S001', 'E001'),   \\
        ('S002', 'E002'),   \\
        ('S003', 'E003'),   \\
        ('S004', 'E004'),   \\
        ('S005', 'E005');

    \item -- Create student\_has table  \\
    CREATE TABLE student\_has (  \\
        roll VARCHAR(20),   \\
        rid VARCHAR(20),    \\
        PRIMARY KEY (roll, rid),    \\
        FOREIGN KEY (roll) REFERENCES student(roll),    \\
        FOREIGN KEY (rid) REFERENCES role(rid)  \\
    ); \vspace{2pt}
    
    -- Insert sample data   \\
    INSERT INTO student\_has (roll, rid) VALUES \\
        ('S001', 'R001'),   \\
        ('S002', 'R002'),   \\
        ('S003', 'R003'),   \\
        ('S004', 'R004'),   \\
        ('S005', 'R005');
\end{itemize}

\newpage

\subsection{\text{Queries \& Responses}}

%A table of queries and responses, left column should contain the query and right column should contain the response.

\begin{itemize}
    \item -- Query 1 \\
    SELECT Student.Roll, Student.Name   \\
    FROM Student    \\
    JOIN MANAGE ON Student.Roll = MANAGE.Roll   \\
    JOIN Event ON MANAGE.EID = Event.EID    \\
    WHERE Event.EName = 'Megaevent';    \vspace{2pt}

    -- Relational Algebra \\
    $\pi_{Student.Roll, Student.Name}(\sigma_{Event.EName = 'Megaevent'}(Student \bowtie MANAGE \bowtie Event))$ \vspace{2pt}
    
    -- Response 1 \\
    \begin{tabular}{|c|c|}
        \hline
        roll & name \\
        \hline
        S001 & Alice Johnson \\
        \hline
    \end{tabular}

    \item -- Query 2 \\
    SELECT Student.Roll, Student.Name   \\
    FROM Student    \\
    JOIN MANAGE ON Student.Roll = MANAGE.Roll   \\
    JOIN Event ON MANAGE.EID = Event.EID    \\
    JOIN Role ON MANAGE.Roll = Role.RID \\
    WHERE Event.EName = 'Megaevent' AND Role.Rname = 'Secretary'; \vspace{2pt}

    -- Relational Algebra \\
    $\pi_{Student.Roll, Student.Name}(\sigma_{Event.EName = 'Megaevent' \wedge Role.Rname = 'Secretary'}(Student \bowtie MANAGE \bowtie Event \bowtie Role))$ \vspace{2pt}
    
    -- Response 2 \\
    \begin{tabular}{|c|c|}
        \hline
        roll & name \\
        \hline
    \end{tabular} \\
    (0 rows)

    \item -- Query 3 \\
    SELECT Participant.Name \\
    FROM Participant    \\
    JOIN Student\_FROM ON Participant.PID = Student\_FROM.PID \\
    JOIN College ON Student\_FROM.College\_Name = College.Name    \\
    JOIN Participant\_HAS ON Participant.PID = Participant\_HAS.PID   \\
    JOIN Event ON Participant\_HAS.EID = Event.EID   \\
    WHERE College.Name = 'IIT Bombay' AND Event.EName = 'Megaevent'; \vspace{2pt}

    -- Relational Algebra \\
    $\pi_{Participant.Name}(\sigma_{College.Name = 'IIT Bombay' \wedge Event.EName = 'Megaevent'}(Participant \bowtie Student\_FROM \bowtie College \bowtie Participant\_HAS \bowtie Event))$ \vspace{2pt}
    
    -- Response 3 \\
    \begin{tabular}{|c|}
        \hline
        name    \\
        \hline
    \end{tabular} \\
    (0 rows)

    \item -- Query 4 \\
    SELECT DISTINCT College.Name    \\
    FROM College    \\
    JOIN Student\_FROM ON College.Name = Student\_FROM.College\_Name   \\
    JOIN Participant\_HAS ON Student\_FROM.PID = Participant\_HAS.PID  \\
    JOIN Event ON Participant\_HAS.EID = Event.EID   \\
    WHERE Event.EName = 'Megaevent';    \vspace{2pt}

    -- Relational Algebra \\
    $\pi_{College.Name}(\sigma_{Event.EName = 'Megaevent'}(College \bowtie Student\_FROM \bowtie Participant\_HAS \bowtie Event))$ \vspace{2pt}
    
    -- Response 4 \\
    \begin{tabular}{|c|}
        \hline
        name \\
        \hline
        IITB \\
        \hline
    \end{tabular}

    \newpage

    \item -- Query 5 \\
    SELECT DISTINCT Event.EName \\
    FROM Event  \\
    JOIN MANAGE ON Event.EID = MANAGE.EID   \\
    JOIN Student ON MANAGE.Roll = Student.Roll  \\
    JOIN Role ON MANAGE.Roll = Role.RID \\
    WHERE Role.Rname = 'Secretary';     \vspace{2pt}

    -- Relational Algebra \\
    $\pi_{Event.EName}(\sigma_{Role.Rname = 'Secretary'}(Event \bowtie MANAGE \bowtie Student \bowtie Role))$ \vspace{2pt}

    -- Response 5 \\
    \begin{tabular}{|c|}
        \hline
        ename \\
        \hline
    \end{tabular} \\
    (0 rows)

    \item -- Query 6 \\
    SELECT DISTINCT Student.Name    \\
    FROM Student    \\
    JOIN Student\_HAS ON Student.Roll = Student\_HAS.Roll \\
    JOIN Role ON Student\_HAS.RID = Role.RID \\
    JOIN Volunteer ON Student.Roll = Volunteer.Roll \\
    JOIN Volunteer\_HAS ON Volunteer.Roll = Volunteer\_HAS.Roll   \\
    JOIN Event ON Volunteer\_HAS.EID = Event.EID \\
    WHERE Student.Dept = 'CSE' AND Event.EName = 'Megaevent';   \vspace{2pt}

    -- Relational Algebra \\
    $\pi_{Student.Name}(\sigma_{Student.Dept = 'CSE' \wedge Event.EName = 'Megaevent'}(Student \bowtie Student\_HAS \bowtie Role \bowtie Volunteer \bowtie \\ Volunteer\_HAS \bowtie Event))$ \vspace{2pt}

    -- Response 6 \\
    \begin{tabular}{|c|}
        \hline
        name \\
        \hline
    \end{tabular} \\
    (0 rows)

    \item -- Query 7 \\
    SELECT DISTINCT Event.EName \\
    FROM Event  \\
    JOIN Volunteer\_HAS ON Event.EID = Volunteer\_HAS.EID \\
    JOIN Volunteer ON Volunteer\_HAS.Roll = Volunteer.Roll   \\
    JOIN Student ON Volunteer.Roll = Student.Roll   \\
    WHERE Student.Dept = 'CSE'; \vspace{2pt}

    -- Relational Algebra \\
    $\pi_{Event.EName}(\sigma_{Student.Dept = 'CSE'}(Event \bowtie Volunteer\_HAS \bowtie Volunteer \bowtie Student))$ \vspace{2pt}

    -- Response 7 \\
    \begin{tabular}{|c|}
        \hline
        ename \\
        \hline
    \end{tabular} \\
    (0 rows)

    \item -- Query 8 \\
    SELECT College.Name \\
    FROM College    \\
    JOIN Student\_FROM ON College.Name = Student\_FROM.College\_Name   \\
    JOIN Participant\_HAS ON Student\_FROM.PID = Participant\_HAS.PID  \\
    JOIN Event ON Participant\_HAS.EID = Event.EID   \\
    WHERE Event.EName = 'Megaevent' \\
    GROUP BY College.Name   \\
    ORDER BY COUNT(Participant\_HAS.PID) DESC    \\
    LIMIT 1;    \vspace{2pt}

    -- Relational Algebra \\
    $\pi_{College.Name}(\sigma_{Event.EName = 'Megaevent'}(College \bowtie Student\_FROM \bowtie Participant\_HAS \bowtie Event))$ \vspace{2pt}

    -- Response 8 \\
    \begin{tabular}{|c|}
        \hline
        name \\
        \hline
        IITB \\
        \hline
    \end{tabular}

    \newpage

    \item -- Query 9 \\
    SELECT College.Name \\
    FROM College    \\
    JOIN Student\_FROM ON College.Name = Student\_FROM.College\_Name   \\
    JOIN Participant\_HAS ON Student\_FROM.PID = Participant\_HAS.PID  \\
    GROUP BY College.Name   \\
    ORDER BY COUNT(Participant\_HAS.PID) DESC    \\
    LIMIT 1;    \vspace{2pt}

    -- Relational Algebra \\
    $\pi_{College.Name}(College \bowtie Student\_FROM \bowtie Participant\_HAS)$ \vspace{2pt}

    -- Response 9 \\
    \begin{tabular}{|c|}
        \hline
        name \\
        \hline
        Harvard \\
        \hline
    \end{tabular}

    \item -- Query 10 \\
    SELECT Student.Dept \\
    FROM Student    \\
    JOIN Student\_HAS ON Student.Roll = Student\_HAS.Roll \\
    JOIN Volunteer ON Student.Roll = Volunteer.Roll \\
    JOIN Volunteer\_HAS ON Volunteer.Roll = Volunteer\_HAS.Roll   \\
    JOIN Event ON Volunteer\_HAS.EID = Event.EID \\
    WHERE Event.EID IN (    \\
        SELECT DISTINCT Participant\_HAS.EID \\
        FROM Participant\_HAS    \\
        JOIN Student\_FROM ON Participant\_HAS.PID = Student\_FROM.PID \\
        WHERE Student\_FROM.College\_Name = 'IITB'    \\
    ) \\
    GROUP BY Student.Dept   \\
    ORDER BY COUNT(Volunteer\_HAS.Roll) DESC \\
    LIMIT 1;    \vspace{2pt}

    -- Relational Algebra \\
    $\pi_{Student.Dept}(\sigma_{Event.EID \in (\pi_{Participant\_HAS.EID}(\sigma_{Student\_FROM.College\_Name = 'IITB'}(Participant\_HAS \bowtie \\ Student\_FROM)))}(Student \bowtie Student\_HAS \bowtie Volunteer \bowtie Volunteer\_HAS \bowtie Event))$ \vspace{2pt}

    -- Response 10 \\
    \begin{tabular}{|c|}
        \hline
        dept \\
        \hline
    \end{tabular} \\
    (0 rows)
\end{itemize}

\begin{flushleft}
    \textit{Note: Some of the Queries do not have equivalent Relational Algebra expressions. In these cases, the closest possible expression has been provided.}
\end{flushleft}

\end{document}
